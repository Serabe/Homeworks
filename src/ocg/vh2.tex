\documentclass[12pt, oneside, a4paper]{article}
\usepackage{graphicx}
\usepackage{ifpdf}
\usepackage{amsfonts}
\usepackage[utf8]{inputenc}
\usepackage{algorithmic}
\usepackage{hyperref}
\def\figureautorefname{figura}
\begin{document}
\title{OCG Hoja 1}\author{Sergio Arbeo}\date{Abril, 2010}\maketitle
\paragraph{2.}\label{ssub:1_} % (fold)
En el campus de una universidad se ubican el rectorado y seis
facultades. La totalidad del correo postal dirigido a dicha
universidad se recibe en el rectorado, y desde ahí ha de distribuirse
a las facultades hasta otras. Para ello, se ha llegado a un acuerdo
con el personal laboral, el cual se encargará de los repartos a cambio
de una remuneración económica por cada desplazamiento realizado. La
cuantía de dichas remuneraciones es independiente de la cantidad de
correspondencia transportada en los desplazamientos, y se muestra en
la siguiente tabla (expresada en euros, ``--'' indica que no se
dispone de personal para realizar el desplazamiento correspondiente):
\\
\\

\begin{tabular}{| c | c | c | c | c | c | c | }
\hline
          & Fac. 1 & Fac. 2 & Fac. 3 & Fac. 4 & Fac. 5 & Fac. 6\\
\hline
Rectorado &   7    &   5    &   5    &   6    &   2    &   4   \\ 
\hline
Fac. 1    &  --    &   3    &  --    &  --    &  --    &   3   \\
\hline
Fac. 2    &  --    &  --    &   2    &  --    &   5    &  --   \\
\hline
Fac. 3    &  --    &  --    &  --    &   3    &  --    &  --   \\
\hline
Fac. 4    &   6    &   1    &  --    &  --    &   3    &  --   \\
\hline
Fac. 5    &  --    &  --    &  --    &   6    &  --    &  --   \\
\hline
Fac. 6    &   2    &  --    &   3    &  --    &  --    &  --   \\
\hline
\end{tabular}
\\
\\

Sabiendo que cada día se recibe correspondencia para todas las
facultades, determinar aplicando el algoritmo de Teoría de Grafos más
apropiado de qué forma deberán realizarse los repartos con objeto de
minimizar el coste total de las remuneraciones.


\subparagraph{Solución.\\}

\begin{figure}
\centering
\includegraphics[height=8cm]{vh2_01.png}
\caption{$R$}
\label{fig:vh2-01}
\end{figure}
En primer lugar, se puede ver la red original en \autoref{fig:vh2-01}.
Ponemos $R^0=R$ y $k=0$. Calculamos $T_{R^0}$, el cual se puede
observar en \autoref{fig:vh2-02}. Como se puede ver, hay dos
ciclos.

\begin{figure}
\centering
\includegraphics[height=3cm]{vh2_02.png}
\caption{$T_{R^{0}}$}
\label{fig:vh2-02}
\end{figure}

Llamaremos $\mu^1$ al ciclo formado por las facultades 2, 3 y
4.

Actualizamos $k=1$ y creamos $R^1=R^0_{\mu^1}$ el cual se puede ver
en \autoref{fig:vh2-03}. Los cálculos se pueden observar en la misma
figura.

\begin{figure}
\centering
\includegraphics[width=8cm]{vh2_03.png}
\caption{$R^{1}$}
\label{fig:vh2-03}
\end{figure}

Creamos $T_{R^{1}}$, el cual podemos ver en \autoref{fig:vh2-04}.

\begin{figure}
\centering
\includegraphics[height=2cm]{vh2_04.png}
\caption{$T_{R^{1}}$}
\label{fig:vh2-04}
\end{figure}

Llamaremos $\mu^2$ al ciclo formado por las facultades 1 y 6.

Actualizamos $k=2$ y creamos $R^2=R^1_{\mu^2}$, el cual se puede ver
en \autoref{fig:vh2-05}. Los cálculos se pueden observar en la misma figura.

\begin{figure}
\centering
\includegraphics[height=2.5cm]{vh2_05.png}
\caption{$R^2$}
\label{fig:vh2-05}
\end{figure}

Como se puede observar en \autoref{fig:vh2-06}, $T_{R^{2}}$ no contiene
circuito alguno. Por lo tanto, ponemos $T^{*}_{\mu^{2}}=T_{R^{2}}$.

\begin{figure}
\centering
\includegraphics[width=3cm]{vh2_06.png}
\caption{$T_{R^2}$}
\label{fig:vh2-06}
\end{figure}

Consideramos $T^{*}_{\mu^{1}}$ la expansión de $T^{*}_{\mu^{2}}$ que
podemos ver en \autoref{fig:vh2-07}.

\begin{figure}
\centering
\includegraphics[width=4cm]{vh2_07.png}
\caption{$T^{*}_{\mu^1}$}
\label{fig:vh2-07}
\end{figure}

Actualizamos $k=1$ y consideramos $T^{*}_{\mu^{0}}$ la expansión de
$T^{*}_{\mu^{1}}$. Dicha expansión se puede encontran en
\autoref{fig:vh2-08}.

\begin{figure}
\centering
\includegraphics[width=4cm]{vh2_08.png}
\caption{$T^{*}_{\mu^0}$}
\label{fig:vh2-08}
\end{figure}
\end{document}

Actualizamos $k=0$ y paramos. Hemos encontrado una arborescencia
soporte de mínimo peso (el peso es 15).
