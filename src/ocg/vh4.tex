\documentclass[12pt, oneside, a4paper]{article}
\usepackage{graphicx}
\usepackage{ifpdf}
\usepackage{amsfonts}
\usepackage[utf8]{inputenc}
\usepackage{algorithmic}
\usepackage{hyperref}
\def\figureautorefname{figura}
\begin{document}
\title{OCG Hoja voluntaria}\author{Sergio Arbeo}\date{Abril, 2010}\maketitle
\paragraph{4.}\label{ssub:1_} % (fold)
Considérese el siguiente digrafo $G=(V,U)$

\begin{figure}[h]
\centering
\includegraphics[height=3cm]{vh4_01.png}
\caption{Grafo original}
\label{fig:vh4-01}
\end{figure}

Para cada arco $u\in U$, la siguiente tabla contiene la mínima
cantidad de flujo $b(u)$ que ha de pasar por $u$, la máxima cantidad
de flujo $c(u)$ que puede pasar por $u$ y el coste $d(u)$ de enviar
cada unidad de flujo a través de $u$:\\


\begin{tabular}{| c | c | c | c |}
\hline
$u$   & $b(u)$ & $c(u)$ & $d(u)$ \\
\hline
(1,2) &   0    &   4    &   5    \\
(1,3) &   1    &   5    &   4    \\
(2,4) &   1    &   4    &   2    \\
(3,5) &   1    &   5    &   0    \\
(4,6) &   2    &   6    &   2    \\
(5,2) &   0    &   2    &   3    \\
(5,4) &   1    &   3    &   1    \\
(5,6) &   0    &   3    &   4    \\
\hline
\end{tabular}\\


Aplicar el algoritmo de Teoría de Grafos más apropiado para determinar
un flujo compatible en $G$ desde 1 hasta 6 con valor $\varphi_{0}=7$ de
mínimo coste, tomando como flujo inicial el nulo (calcular el coste
del flujo obtenido).

\subparagraph{Solución.\\}

Para hallar el flujo compatible en $G$ de mínimo coste aplicaremos el
algoritmo de Busacker-Gowen-Klein. Para ello, primero se ha de usar el algoritmo de
Herz para encontrar un flujo compatible $\varphi^*$ en $G$.

En primer lugar, etiquetamos a partir del arco $(1,3)$ los arcos
$(3,5)$, $(5,6)$ y $(6,1)$. Como aún no es un flujo compatible, etiquetamos a
partir del arco $(2,4)$ los arcos $(4,6)$, $(6,1)$ y
$(1,2)$. Finalmente, etiquetamos a partir del arco $(5,4)$ el $(4,6)$ y
el $(6,5)$, retrocediendo flujo por este último El
resultado se puede observar en \autoref{fig:vh4-02}.

\begin{figure}[h]
\centering
\includegraphics[height=3cm]{vh4_02.png}
\caption{Flujo compatible}
\label{fig:vh4-02}
\end{figure}

Se observa que $\varphi_{0}^{*}=7$. Hecho esto, pasamos al paso dos del algoritmo de Busacker-Gowain-Klein. Podemos ver
$R_{\varphi^{*}}$ en \autoref{fig:vh4-03}.

\begin{figure}[h]
\centering
\includegraphics[height=4cm]{vh4_03.png}
\caption{$R_{\varphi^{*}}$}
\label{fig:vh4-03}
\end{figure}

Podemos observar un ciclo de longitud negativa en
$\mu=\{(4,2), (2,1), (1,3), (3,5), (5,4)\}$. $c(\mu)$ sería:

$$
c(\mu)= \min\{
\min\{5-3,5-3,3-2\},
\min\{4-1,4-0\}
\}=1
$$

El nuevo flujo se puede ver en \autoref{fig:vh4-04}.

\begin{figure}[h]
\centering
\includegraphics[height=4cm]{vh4_04.png}
\caption{Nuevo flujo.}
\label{fig:vh4-04}
\end{figure}

Recalculamos $R_{\varphi^{*}}$, y nos queda lo que se observa en
\autoref{fig:vh4-05}.

\begin{figure}[h]
\centering
\includegraphics[height=4cm]{vh4_05.png}
\caption{$R_{\varphi^{*}}$}
\label{fig:vh4-05}
\end{figure}

El ciclo $\mu=\{(6, 4), (4, 2), (2, 1), (1, 3), (3, 5), (5, 6)\}$ tiene
longitud negativa.

$$
c(\mu)= \min\{
\min\{5-4,5-4,3-1\},
\min\{6-2, 3-1, 3-0\}
\}=1
$$

Por lo que el nuevo flujo se puede observar en \autoref{fig:vh4-06}.

\begin{figure}[h]
\centering
\includegraphics[height=4cm]{vh4_06.png}
\caption{Tercer flujo}
\label{fig:vh4-06}
\end{figure}

Recalculamos $R_{\varphi^{*}}$ en \autoref{fig:vh4-07}.

\begin{figure}[h]
\centering
\includegraphics[height=4cm]{vh4_07.png}
\caption{Último $R_{\varphi^{*}}$}
\label{fig:vh4-07}
\end{figure}

Este $R_{\varphi^{*}}$ ya no contiene ciclos de longitud negativa, por
lo tanto, el paso tres del algoritmo nos confirma que el flujo
observado en \autoref{fig:vh4-06} es un flujo compatible de mínimo
coste, en este caso, 55.

\end{document}
