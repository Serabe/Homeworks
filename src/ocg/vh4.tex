\documentclass[12pt, oneside, a4paper]{article}
\usepackage{graphicx}
\usepackage{ifpdf}
\usepackage{amsfonts}
\usepackage[utf8]{inputenc}
\usepackage{algorithmic}
\usepackage{hyperref}
\def\figureautorefname{figura}
\begin{document}
\title{OCG Hoja voluntaria}\author{Sergio Arbeo}\date{Abril, 2010}\maketitle
\paragraph{4.}\label{ssub:1_} % (fold)
Considérese el siguiente digrafo $G=(V,U)$

\begin{figure}
\centering
\includegraphics[height=8cm]{vh4_01.png}
\caption{Grafo original}
\label{fig:vh4-01}
\end{figure}

Para cada arco $u\in U$, la siguiente tabla contiene la mínima
cantidad de flujo $b(u)$ que ha de pasar por $u$, la máxima cantidad
de flujo $c(u)$ que puede pasar por $u$ y el coste $d(u)$ de enviar
cada unidad de flujo a través de $u$:

\begin{tabular}{| c | c | c | c |}
\hline
$u$   & $b(u)$ & $c(u)$ & $d(u)$ \\
\hline
(1,2) &   0    &   4    &   5    \\
(1,3) &   1    &   5    &   4    \\
(2,4) &   1    &   4    &   2    \\
(3,5) &   1    &   5    &   0    \\
(4,6) &   2    &   6    &   2    \\
(5,2) &   0    &   2    &   3    \\
(5,4) &   1    &   3    &   1    \\
(5,6) &   0    &   3    &   4    \\
\hline
\end{tabular}

Aplicar el algoritmo de Teoría de Grafos más apropiado para determinar
un flujo compatible en $G$ desde 1 hasta 6 con valor $\phi_{0}=7$ de
mínimo coste, tomando como flujo inicial el nulo (calcular el coste
del flujo obtenido).

\subparagraph{Solución.\\}



\end{document}
