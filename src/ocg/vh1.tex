\documentclass[12pt, oneside, a4paper]{article}
\usepackage{graphicx}
\usepackage{ifpdf}
\usepackage{amsfonts}
\usepackage[utf8]{inputenc}
\usepackage{algorithmic}
\begin{document}
\title{OCG Hoja 1}\author{Sergio Arbeo}\date{Marzo, 2010}\maketitle
\paragraph{1.}\label{ssub:1_} % (fold)
Sea $G=(V,E)$ un grafo, donde $V=\{1,...,n\}$, $E=\{e_{1},...,e_{m}\}$
y $n,m\in\mathbb{N}$, y sea $W=\{1,...,q\}$, donde $q\in\mathbb{N}$ y
$q<n$. Para cada uno de los casos siguientes, dise\~nar un algoritmo que
determine la incidencia de $W$ y calcular su complejidad.


\subparagraph{(a)} $G$ está expresado mediante su matriz de adyacencia.

Sea $M$ la matriz de adyacencia.

\begin{algorithmic}[1]
\STATE $incidence \gets [false,...,false]$
\FOR{$r$ in $W$}
  \FOR{$s$ in $V\backslash W$}
    \STATE $incidence[s] \gets M[r][s] > 0$
  \ENDFOR
\ENDFOR
\end{algorithmic}

El algoritmo funciona recorriendo primero los nodos contenidos en $W$
y para cada uno de ellos, se recorren los nodos en $V\backslash W$. Si
el valor de la matriz de adyacencia para dichos nodos es mayor que
cero, el nodo de $V\backslash W$ se añade a la incidencia.

El bucle exterior se ejecuta $q$ veces mientras que el interior se
ejecuta $n-q$ veces. El interior de cada bucle toma tiempo constante
(asignación, lectura y comparación son operaciones elementales) por lo
que la complejidad de los bucles es $O(q\cdot (n-q))$. Tomando el
tiempo de inicialización de $incidences$ como constante, se tiene que
éste es la complejidad del algoritmo.

Nótese que la inicialiación de $incidence$ es como mucho $O(n)$, lo cual
es menor que $O(q\cdot (n-q))$.

\subparagraph{(b)} $G$ está expresado mediante su matriz de
incidencia.

Sea $M$ la matriz de incidencia.\\\\

\begin{algorithmic}[1]
\STATE $incidence \gets  [false,...,false]$
\FOR{$i=1$ to $m$}
  \STATE $foundNodes \gets 0$
  \STATE $n \gets 1$
  \FOR{$j$ in $V\backslash W$}
    \IF{$M[i][j]==1$}
      \STATE $n \gets j$
      \STATE $foundNodes \gets foundNodes + 1$
    \ENDIF
  \ENDFOR
  
  \IF{$incidence[n] \gets foundNodes == 1$}
    \FOR{$j$ in $W$}
      \IF{$M[i][j] ==1$}
        \STATE $incidence[n] \gets true$
        \STATE break
      \ENDIF
    \ENDFOR
   \ENDIF
\ENDFOR
\end{algorithmic}

En primer lugar, el algoritmo inicialia un array de booleanos, todos a
false. Después recorre una a una todas las aristas, inicializando las
variables found-nodes a cero y n a uno. Para cada arista, comprueba todos
los nodos que no se encuentran en $W$. Si un vértice es extremo de la
arista entonces n captura el nodo y found-nodes se actualiza sumándole
uno.

Una vez recorridos los nodos de $V\backslash W$, se comprueba que el
número de nodos encontrados es 1 (si se han encontrado dos, la arista
no conecta con ningún nodo en $W$ mientras que si no se ha encontrado
ninguno, dicha arista no conecta con ningún nodo en $V\backslash
W$). Si es así, se comprueba que exista algún nodo en $W$ que sea
extremo de la arista, pues podría ser que la arista fuese un bucle. Si
se encuentra algún nodo, se marca el nodo encontrado en $V\backslash
W$ como que está en la incidencia de $W$ (se asigna su número a $n$) y
se sale del bucle (anidado).

Para calcular la complejidad tenemos que el bucle se ejecuta $m$ veces
y cada ejecución tarda $O(n-q)$ para el primer bucle y $O(q)$ para el
segundo, aunque éste no se ejecute siempre. Por lo tanto, cada
iteración quedaría acotada por $O(\max(n-q, q))$. De esta forma, la
complejidad del bucle queda como $O(m\cdot\max(n-q,q))$. Tomando la
inicialización de $incidence$ como constante quedaría dicho valor como
la complejidad del algoritmo.

\subparagraph{Nota:} se podrían considerar distintas optimizaciones,
como dejar de considerar en los bucles los nodos que hayamos añadido a
la incidencia. Dichas optimizaciones no modificarían la complejidad
expresada en $O$-notación, pues se seguiría teniendo dicha cota para
el peor caso (en el que la incidencia sea nula). En cambio, se podría
considerar una optimización que sí mejoraría el tiempo del segundo
algoritmo:

Redefínase la matriz de incidencia de forma que si una arista es un
bucle, el valor sea 2 en vez de 1. Así sólo habría que comprobar que
existe un único nodo en el conjunto más pequeño (obsérvese que la
condición de los if comprueban la igualdad estricta). De esta forma,
el tiempo vendría acotado por $O(m\cdot\min(n-q,q))$.

\end{document}
