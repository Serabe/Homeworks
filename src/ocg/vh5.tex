\documentclass[12pt, oneside, a4paper]{article}
\usepackage{graphicx}
\usepackage{ifpdf}
\usepackage{amsfonts}
\usepackage[utf8]{inputenc}
\usepackage{algorithmic}
\usepackage{hyperref}
\def\figureautorefname{figura}
\begin{document}
\title{OCG Hoja voluntaria}\author{Sergio Arbeo}\date{Mayo, 2010}\maketitle
\paragraph{5.}\label{ssub:1_} % (fold)
Considérese el siguiente grafo $G=(V,E)$:

\begin{figure}[h]
\centering
\includegraphics[height=5cm]{vh5_01.png}
\caption{Grafo original}
\label{fig:vh4-01}
\end{figure}


\subparagraph{(a)} Aplicar a $G$ el algoritmo de determinación de un emparejamiento
maximal eligiendo siempre el menor vértice que satisfaga las
condiciones requeridas.

\subparagraph{Solución.\\}


{\bf Paso 1.}
$V'={1,2,3,4,5,6,7,8,9,10,11,12,13}$,
$M=\emptyset$.

{\bf Paso 2.} $1\in V'$,
$V'=V'\backslash
\{1\}$. $\Gamma(1)\cap V'=\{3, 4\}\neq
\emptyset$. Elegimos
$3\in\Gamma(1)\cap V'$ y ponemos
$V'=V'\backslash \{3\}$, $M=M\cup\{\{1,3\}\}$.

{\bf Paso 3.} Como $|V'|\not\leq 1$
volvemos al paso 2.

{\bf Paso 2.} $2\in V'$, $V'=V'\backslash
\{2\}$. $\Gamma(2)\cap V'=\{4, 5, 6,
10\}\neq\emptyset$. Elegimos
$4\in\Gamma(2)\cap V'$ y ponemos
$V'=V'\backslash \{4\}$,
$M=M\cup\{\{2,4\}\}$.

{\bf Paso 3.} Como $|V'|\not\leq 1$
volvemos al paso 2.

{\bf Paso 2.} $5\in V'$, $V'=V'\backslash
\{5\}$. $\Gamma(5)\cap V'=\{8, 9, 10,
12\}\neq\emptyset$. Elegimos
$8\in\Gamma(5)\cap V'$ y ponemos
$V'=V'\backslash \{8\}$,
$M=M\cup\{\{5,8\}\}$.

{\bf Paso 3.} Como $|V'|\not\leq 1$
volvemos al paso 2.

{\bf Paso 2.} $6\in V'$, $V'=V'\backslash
\{6\}$. $\Gamma(6)\cap
V'=\{10\}\neq\emptyset$. Elegimos
$10\in\Gamma(6)\cap V'$ y ponemos
$V'=V'\backslash \{10\}$,
$M=M\cup\{\{6,10\}\}$.

{\bf Paso 3.} Como $|V'|\not\leq 1$
volvemos al paso 2.

{\bf Paso 2.} $7\in V'$, $V'=V'\backslash
\{7\}$. $\Gamma(7)\cap
V'=\emptyset$.

{\bf Paso 3.} Como $|V'|\not\leq 1$
volvemos al paso 2.

{\bf Paso 2.} $9\in V'$, $V'=V'\backslash
\{9\}$. $\Gamma(9)\cap
V'=\emptyset$.

{\bf Paso 3.} Como $|V'|\not\leq 1$
volvemos al paso 2.

{\bf Paso 2.} $11\in V'$, $V'=V'\backslash
\{11\}$. $\Gamma(11)\cap
V'=\emptyset$.

{\bf Paso 3.} Como $|V'|\not\leq 1$
volvemos al paso 2.

{\bf Paso 2.} $12\in V'$, $V'=V'\backslash
\{12\}$. $\Gamma(12)\cap
V'=\emptyset$.

{\bf Paso 3.} Como $|V'|\leq 1$,
hemnos terminado.

$M=\{\{1,3\},\{2,4\},\{5,8\},\{6,10\}\}$
es un emparejamiento maximal.

\subparagraph{(b)} Tomando como emparejamiento inicial de $G$ el obtenido en el
apartado anterior, determinar el emparejamiento de $G$ de máximo
cardinal aplicando el algoritmo de Teoría de Grafos más apropiado,
eligiendo para cada paso el menor vértice que satisfaga las
condiciones requeridas.

\subparagraph{Solución.\\}


\end{document}
