\documentclass[12pt, oneside, a4paper]{article}
\usepackage{graphicx}
\usepackage{ifpdf}
\usepackage{amsfonts}
\usepackage[utf8]{inputenc}
\usepackage{algorithmic}
\usepackage{hyperref}
\def\figureautorefname{figura}
\begin{document}
\title{OCG Hoja voluntaria}\author{Sergio Arbeo}\date{Mayo, 2010}\maketitle
\paragraph{5.}\label{ssub:1_} % (fold)
Considérese el siguiente grafo $G=(V,E)$:

\begin{figure}[h]
\centering
\includegraphics[height=5cm]{vh5_01.png}
\caption{Grafo original}
\label{fig:vh4-01}
\end{figure}


\subparagraph{(a)} Aplicar a $G$ el algoritmo de determinación de un emparejamiento
maximal eligiendo siempre el menor vértice que satisfaga las
condiciones requeridas.

\subparagraph{Solución.\\}


{\bf Paso 1.}
$V'={1,2,3,4,5,6,7,8,9,10,11,12,13}$,
$M=\emptyset$.

{\bf Paso 2.} $1\in V'$,
$V'=V'\backslash
\{1\}$. $\Gamma(1)\cap V'=\{3, 4\}\neq
\emptyset$. Elegimos
$3\in\Gamma(1)\cap V'$ y ponemos
$V'=V'\backslash \{3\}$, $M=M\cup\{\{1,3\}\}$.

{\bf Paso 3.} Como $|V'|\not\leq 1$
volvemos al paso 2.

{\bf Paso 2.} $2\in V'$, $V'=V'\backslash
\{2\}$. $\Gamma(2)\cap V'=\{4, 5, 6,
10\}\neq\emptyset$. Elegimos
$4\in\Gamma(2)\cap V'$ y ponemos
$V'=V'\backslash \{4\}$,
$M=M\cup\{\{2,4\}\}$.

{\bf Paso 3.} Como $|V'|\not\leq 1$
volvemos al paso 2.

{\bf Paso 2.} $5\in V'$, $V'=V'\backslash
\{5\}$. $\Gamma(5)\cap V'=\{8, 9, 10,
12\}\neq\emptyset$. Elegimos
$8\in\Gamma(5)\cap V'$ y ponemos
$V'=V'\backslash \{8\}$,
$M=M\cup\{\{5,8\}\}$.

{\bf Paso 3.} Como $|V'|\not\leq 1$
volvemos al paso 2.

{\bf Paso 2.} $6\in V'$, $V'=V'\backslash
\{6\}$. $\Gamma(6)\cap
V'=\{10\}\neq\emptyset$. Elegimos
$10\in\Gamma(6)\cap V'$ y ponemos
$V'=V'\backslash \{10\}$,
$M=M\cup\{\{6,10\}\}$.

{\bf Paso 3.} Como $|V'|\not\leq 1$
volvemos al paso 2.

{\bf Paso 2.} $7\in V'$, $V'=V'\backslash
\{7\}$. $\Gamma(7)\cap
V'=\emptyset$.

{\bf Paso 3.} Como $|V'|\not\leq 1$
volvemos al paso 2.

{\bf Paso 2.} $9\in V'$, $V'=V'\backslash
\{9\}$. $\Gamma(9)\cap
V'=\emptyset$.

{\bf Paso 3.} Como $|V'|\not\leq 1$
volvemos al paso 2.

{\bf Paso 2.} $11\in V'$, $V'=V'\backslash
\{11\}$. $\Gamma(11)\cap
V'=\emptyset$.

{\bf Paso 3.} Como $|V'|\not\leq 1$
volvemos al paso 2.

{\bf Paso 2.} $12\in V'$, $V'=V'\backslash
\{12\}$. $\Gamma(12)\cap
V'=\emptyset$.

{\bf Paso 3.} Como $|V'|\leq 1$,
hemnos terminado.

$M=\{\{1,3\},\{2,4\},\{5,8\},\{6,10\}\}$
es un emparejamiento maximal.

\subparagraph{(b)} Tomando como emparejamiento inicial de $G$ el obtenido en el
apartado anterior, determinar el emparejamiento de $G$ de máximo
cardinal aplicando el algoritmo de Teoría de Grafos más apropiado,
eligiendo para cada paso el menor vértice que satisfaga las
condiciones requeridas.

\subparagraph{Solución.\\}

Denotaremos los pasos por 8.\{2,3\}.x siendo 8.y el algoritmo y x el
número de paso.\\

{\bf Paso 8.3.1.} Cogemos $M=\{\{1,3\},\{2,4\},\{5,8\},\{6,10\}\}$ y $R=\{7, 9, 11, 12, 13\}$.\\

{\bf Paso 8.3.2.} Como $|R|\not\leq 1$ escogemos $r=7$ y actualizamos
$R=\{9, 11, 12, 13\}$.\\

{\bf Paso 8.3.3.} Pasamos a aplicar el algoritmo 8.2.\\

{\bf Paso 8.2.1.} Ponemos $G^{0}=G$, $M^{0}=M$, $V^{0}=\{7\}$,
$E^{0}=\emptyset$, $k=0$ y $q(v)=0$, $\forall v\in V$.\\

{\bf Paso 8.2.2.} El único vértice en $V^{0}$ es $v^{*}=7$ y cumple las
características ($a(7)=0$ y $q(7)=0$). Ponemos $q(7)=1$.\\

{\bf Paso 8.2.3.} Como $\not\exists v\in\Gamma(v^{*})$ no saturado por
$M^{0}$, ponemos $V'=\Gamma(7)\\{v\in V^{0}|\{v*,v\}\in
  E^{0}\}=\{3,4\}$ (a partir de ahora pondré el resultado directo).\\

{\bf Paso 8.2.4.} $V'\neq \emptyset$ por lo que cogemos $v=3\in V'$ y
actualizamos $V'=\{4\}$.\\

{\bf Paso 8.2.5.} $3\not\in V^{0}$\\

{\bf Paso 8.2.6.} $3\not\in V^{0}$, por lo que ponemos $V^{0}=\{1,3,7\}$ y
$E^{0}=\{\{7,3\}, \{1,3\}\}$. Ir al paso 8.2.4.\\

{\bf Paso 8.2.4.} Puesto que $V'\neq \emptyset$, cogemos $v=4\in V'$ y
actualizamos $V'=\emptyset$.\\

{\bf Paso 8.2.5.} $4\not\in V^{0}$\\

{\bf Paso 8.2.6.} $4\not\in V^{0}$, por lo que ponemos
$V^{0}=\{1,2,3,4,7\}$ y $E^{0}=E^{0}\cup\{\{4,7\},\{2,4\}\}$. Vamos al
paso 8.2.4.\\

{\bf Paso 8.2.4.} Puesto que $V'=\emptyset$, vamos al paso 8.2.2.\\

{\bf Paso 8.2.2.} Para $v^{*}=1$ se verifica que $a(1)=2$ es par y
$q(1)=0$. Ponemos $q(1)=1$.\\

{\bf Paso 8.2.3.} Puesto que no existe un vértice no saturado por
$M^{0}$ en la incidencia de 1, ponemos $V'=\{4\}$\\

% SUPOSICIÓN
{\bf Paso 8.2.4.} $V'\neq\emptyset$, por lo que ponemos $v=4$,
$V'=\emptyset$.\\

{\bf Paso 8.2.5.} $4\in V^{0}$, pero $a(4)$ no es par. Vamos al paso
8.2.2.\\

{\bf Paso 8.2.2.} $v^{*}=2$ verifica que $a(2)=2$ es par y que
$q(2)=0$. Ponemos $q(2)=1$.\\

{\bf Paso 8.2.3.} $\not\exists v\in\Gamma(2)$ no saturado por $M^{0}$,
por lo que ponemos $V'=\{5,6,10\}$.\\

{\bf Paso 8.2.4.} Cogemos $v=5\in V'$ y ponemos $V'=\{6,10\}$.\\

{\bf Paso 8.2.5.} $5\not\in V^{0}$\\

{\bf Paso 8.2.6.} $5\not\in V^{0}$. Actualizamos
$V'=\{1,2,3,4,5,7,8\}$ y $E^{0}=E^{0}\cup\{\{5,8\},\{2,5\}\}$. Vamos
al paso 8.2.4.\\

{\bf Paso 8.2.4.} Puesto que $V'\neq\emptyset$, cogemos $v=6\in V'$ y
actualizamos $V'=\{10\}$.\\

{\bf Paso 8.2.5.} $6\not\in V^{0}$\\

{\bf Paso 8.2.6.} $6\not\in V^{0}$, por lo que actualizamos
$V^{0}=\{1,2,3,4,5,6,7,8,10\}$ y
$E^{0}=E^{0}\cup\{\{2,6\},\{6,10\}\}$. Volvemos al paso 8.2.4.\\

{\bf Paso 8.2.4.} Cogemos $v=10\in V'$ y actualizamos $V'=\emptyset$.\\

{\bf Paso 8.2.5.} Puesto que $10\in V^{0}$ y $a(10)=4$, por lo que es
par. Actualizamos $E^{0}=E^{0}\cup\{\{2,10\}\}$, $k=1$. El único ciclo
contenido en $E^{0}$ es el formado por los vértices
$W^{1}=\{2,6,10\}$. Restringimos $G^{1}=G^{0}/W^{1}$, $M^{1}$ es como
$M^{0}$ eliminando $\{6,10\}$ (pues ambos se transforman en el nodo
14) y modificando $\{2,4\}$ por
$\{4,14\}$. $V^{1}=\{1,3,4,5,7,8,14\}$,
$E^{1}=\{\{3,7\},\{1,3\},\{4,7\},\{4,14\},\{5,14\},\{4,8\}\}$  y
$q(14)=0$. Volvemos al paso 8.2.2.\\

{\bf Paso 8.2.2.} $v=14\in V^{1}$ y $a(14)=2<4=a(8)$. Ponemos
$q(14)=1$.\\

{\bf Paso 8.2.3.} $9\in\Gamma(14)$ (pues 10 estaba conectado con 9) y
9 no está saturado por $M^{1}$, por lo que se determina una cadena
incremental. Dicha cadena incremental es $7, 4, 2, 6, 10, 9$. Vamos al
paso 8.3.3.\\

{\bf Paso 8.3.3.} Aplicamos la cadena incremental anterior a $M$ y
resulta en\\

$M = \{\{1,3\},\{2,6\},\{4,7\},\{5,8\},\{9,10\}\}$.\\

Actualizamos $R=\{11,12,13\}$ y volvemos al paso 8.3.2.\\

{\bf Paso 8.3.2} Cogemos $r=11$ y ponemos $R=\{12,13\}$. Vamos al paso
8.2.1.\\

{\bf Paso 8.2.1.} Ponemos $G^{0}=G$, $M^{0}=M$, $V^{0}=\{11\}$,
$E^{0}=\emptyset$, $k=0$ y $q(v)=0$, $\forall v\inV$.\\

{\bf Paso 8.2.2.} Cogemos $v^{*}=11$ y ponemos $q(11)=1$.

{\bf Paso 8.2.3.} El algoritmo nos obliga a poner $V'=\{4\}$.

{\bf Paso 8.2.4.} Cogemos $v=4$, actualizamos $V'=\emptyset$.

{\bf Paso 8.2.5.} No se cumplen las condiciones de este paso.

{\bf Paso 8.2.6.} Actualizamos $V^{0}=\{4,7,11\}$ y
$E^{0}=\{\{4,11\},\{7,11\}\}$. Volvemos al paso 4.\\

{\bf Paso 8.2.4.} Volvemos al paso 8.2.2.\\

{\bf Paso 8.2.2.} Cogemos $v^{*}=7$ y ponemos $q(7)=7$.\\

{\bf Paso 8.2.3.} Ponemos $V'=\{3\}$.\\

{\bf Paso 8.2.4.} Ponemos $v=3$ y actualizamos $V'=\emptyset$.\\

{\bf Paso 8.2.5.} $3\not\in V^{0}$.\\

{\bf Paso 8.2.6.} Actualizamos $V^{0}=\{1,3,4,7,11\}$ y
$E^{0}=E^{0}\cup\{\{3,7\},\{1,3\}\}$. Volvemos al paso 8.2.4.\\

{\bf Paso 8.2.4.} Puesto que $V'=\emptyset$, volvemos al paso 8.2.2.\\

{\bf Paso 8.2.2.} Cogemos $v^{*}=1$ y ponemos $q(1)=1$.\\

{\bf Paso 8.2.3.} Inicializamos $V'=\{4\}$.\\

{\bf Paso 8.2.4.} Ponemos $v=4$, $V'=\emptyset$.\\

{\bf Paso 8.2.5.} $4\in V$, pero $a(4)$ es impar. Volvemos al paso
8.2.2.\\

{\bf Paso 8.2.2.} $\not\exists v\in V^{0}$ tal que $a(v)$ sea par y
$q(v)=0$. Por lo tanto no existe cadena incremental tal que 11 sea uno
de sus extremos. Volvemos al paso 8.3.2.\\

{\bf Paso 8.3.2.} Puesto que $R=\{12, 13\}$, cogemos $r=12$ y dejamos
$R=\{13\}$. Reiniciamos el algoritmo $8.2.$.\\

{\bf Paso 8.2.1.} Inicializamos $G^{0}=G$, $M^{0}=M$, $V^{0}=\{12\}$,
$E^{0}=\emptyset$, $k=0$ y $q(v)=0$, $\forall v\in V$.\\

{\bf Paso 8.2.2.} Cogemos $v^{*}=12$ y ponemos $q(12)=1$.\\

{\bf Paso 8.2.3.} Ponemos $V'=\{5\}$.\\

{\bf Paso 8.2.4.} Cogemos $v=5$, $V'=\emptyset$.\\

{\bf Paso 8.2.5.} $5\not\in V^{0}$.\\

{\bf Paso 8.2.6.} Actualizamos $V^{0}=\{5,8,12\}$ y
$E^{0}=\{\{5,12\},\{5,8\}\}$. Volvemos al paso 8.2.4.\\

{\bf Paso 8.2.4.} Volvemos al paso 8.2.2.\\

{\bf Paso 8.2.2.} Cogemos $v^{*}=8$ y ponemos $q(8)=1$.\\

{\bf Paso 8.2.3.} Ponemos $V'=\{4\}$.\\

{\bf Paso 8.2.4.} Cogemos $v=4$ y actualizamos $V'=\emptyset$.\\

{\bf Paso 8.2.5.} $4\not\in V^{0}$.\\

{\bf Paso 8.2.6.} Actualizamos $V^{0}=\{4,5,7,8,12\}$ y
$E^{0}=E^{0}\cup\{\{4,8\},\{4,7\}\}$. Volvemos al paso 8.2.4.\\

{\bf Paso 8.2.4.} Volvemos al paso 8.2.2 pues $V'=\emptyset$\\

{\bf Paso 8.2.2.} Cogemos $v^{*}=7$ y ponemos $q(7)=1$.\\

{\bf Paso 8.2.3.} Ponemos $V'=\{3\}$.\\

{\bf Paso 8.2.4.} Cogemos $v=3$ y lo quitamos de $V'$, quedando
$V'=\emptyset$\\

{\bf Paso 8.2.5.} $3\not\in V^{0}$.\\

{\bf Paso 8.2.6.} Actualizamos $V^{0}=\{1,3,4,5,7,8,12\}$ y
$E^{0}=E^{0}\cup\{\{3,7\},\{1,3\}\}$. Volvemos al paso 8.2.4.\\

{\bf Paso 8.2.4.} Volvemos al paso 8.2.2.\\

{\bf Paso 8.2.2.} Cogemos $v^{*}=1$ y ponemos $q(1)=1$.\\

{\bf Paso 8.2.3.} Ponemos $V'=\{4\}$.\\

{\bf Paso 8.2.4.} Cogemos $v=4\in V'$ y $V'=\emptyset$.\\

{\bf Paso 8.2.5.} $4\in V^{0}$. Volvemos al paso 2.\\

{\bf Paso 8.2.2.} Volvemos al 8.3.2 pues $\not\exists v\in V^{0}$ tal
que $a(v)$ sea par y $q(v)=0$.\\

{\bf Paso 8.3.2.} Puesto que $|R|=1\leq 1$ paramos. Hemos encontrado
un emparejamiento de máximo cardinal. Éste es
$M=\{\{1,3\},\{2,6\},\{4,7\},\{5,8\},\{9,10\}\}$.\\ 

\end{document}
